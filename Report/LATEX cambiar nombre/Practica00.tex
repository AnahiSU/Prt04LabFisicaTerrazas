%%%%%%%%%%%%%%%%%%%%%%%%%%%%%%%%%%%%%%%%%%%%%%%%%%%%%%%%%%%%%%%%%%%%%%%%%%%%%%%%%%%%
% ****** Start of file apssamp.tex ******
%
%   This file is part of the APS files in the REVTeX 4.2 distribution.
%   Version 4.2a of REVTeX, December 2014
%
%   Copyright (c) 2014 The American Physical Society.
%
%   See the REVTeX 4 README file for restrictions and more information.
%
% TeX'ing this file requires that you have AMS-LaTeX 2.0 installed
% as well as the rest of the prerequisites for REVTeX 4.2
%
% See the REVTeX 4 README file
% It also requires running BibTeX. The commands are as follows:
%
%  1)  latex apssamp.tex
%  2)  bibtex apssamp
%  3)  latex apssamp.tex
%  4)  latex apssamp.tex
%%%%%%%%%%%%%%%%%%%%%%%%%%%%%%%%%%%%%%%%%%%%%%%%%%%%%%%%%%%%%%%%%%%%%%%%%%%%%%%%%%%%
\documentclass[%
 reprint,
%superscriptaddress,
groupedaddress,
unsortedaddress,
%runinaddress,
%frontmatterverbose, 
%preprint,
%preprintnumbers,
%nofootinbib,
%nobibnotes,
%bibnotes,
 amsmath,amssymb,
 aps,
%pra,
%prb,
%rmp,
%prstab,
%prstper
%floatfix,
superscriptaddress
]{revtex4-2}
% Firmato de articulo
\usepackage{graphicx}% Include figure files
\usepackage{dcolumn}% Align table columns on decimal point
\usepackage{bm}% bold math
\usepackage[utf8]{inputenc}
\usepackage[spanish]{babel}

\usepackage{hyperref}% add hypertext capabilities
\usepackage{hyperref}
\hypersetup{
	unicode=false,          % non-Latin characters in Acrobat’s bookmarks
	pdftoolbar=true,        % show Acrobat’s toolbar?
	pdfmenubar=true,        % show Acrobat’s menu?
	pdffitwindow=false,     % window fit to page when opened
	pdfstartview={FitH},    % fits the width of the page to the window
	pdftitle={My title},    % title
	pdfauthor={Author},     % author
	pdfsubject={Subject},   % subject of the document
	pdfcreator={Creator},   % creator of the document
	pdfproducer={Producer}, % producer of the document
	pdfkeywords={keyword1, key2, key3}, % list of keywords
	pdfnewwindow=true,      % links in new PDF window
	colorlinks=true,
	linkcolor=blue,
	filecolor=magenta,      
	urlcolor=red,
	citecolor=blue,        % color of links to bibliography
	pdftitle={Sharelatex Example},
	bookmarks=true,
	pdfpagemode=FullScreen,
}

\usepackage{bm}            % special 'bold-math' 


\usepackage{fancyhdr}
\pagestyle{fancy}
\fancyhf{}
\fancyhead[LE,RO]{Dep. Física}
\fancyhead[RE,LO]{FCyT  - UMSS}

\fancyfoot[CE,CO]{B6}
\fancyfoot[RE,LO]{Lab. Física General}
\fancyfoot[LE,RO]{\thepage}

\renewcommand{\headrulewidth}{0.4pt}
\renewcommand{\footrulewidth}{0.4pt}

%\usepackage[mathlines]{lineno}% Enable numbering of text and display math
%\linenumbers\relax % Commence numbering lines

\begin{document}

\preprint{APS/123-QED}

\title{T\'itulo de la pr\'actica}% Force line breaks with \\
%\thanks{A footnote to the article title}%

\author{Apellido1, Nombre1.}
\affiliation{Departamento de Física }
\email{Primer.Author@umss.edu.bo}

\author{Apellido2, Nombre2.}
\affiliation{Departamento de Quimica}
\author{Apellido3, Nombre3.}
\affiliation{Departamento de ...}
%Lines break automatically or can be forced 

%with \\


\date{\today}% It is always \today, today,
             %  but any date may be explicitly specified

\begin{abstract}
Es un enunciado resumido del  contenido de  la  practica y  organizado secuencialmente  
(menos de $ 150 $ palabras). Establece brevemente el problema y propósito de la práctica. 
Indica el plan teórico o experimental a que se siguió. Resume las principales 
aportaciones y principales conclusiones.\\ \textbf{Importante}:El  resumen  es  la  
información  condensada  que  el  profesor  recibe  de  la  práctica realizada.  Debe  
ser  corto  y  claro  para  que  el  profesor  determine  si  se  entendió  y  realizo 
correctamente  la  práctica.  Incluye  información  de  seguridad  cuando  sea  
necesario.  No  evalúes conclusiones.\\\textbf{Súper  Importante}: Escribe  el  resumen  
al  último  para  estar  seguro  que  refleja  con  exactitud  la práctica realizada
\begin{description}
\item[Palabras clave]: medición, vernier.
\end{description}
\end{abstract}

%\keywords{Suggested keywords}%Use showkeys class option if keyword
                              %display desired
\maketitle

%\tableofcontents

\section{Introducción}
\label{sec:introduccion}
La  introducción  debe  responder  a  la  pregunta  de  "porqué  se  ha  hecho  este  
trabajo".Una  buena introducción,  es  una  oración  clara  del  problema  y  de  las  
razones  por  las  que  lo  estamos estudiando.  Nos  da  una  concisa  y  apropiada  
discusión  del  problema,  su  significado,  alcances  y 
limitaciones.\\\textbf{Importante}:En esta sección se responde a la pregunta de "cómo se 
ha hecho la práctica".\\\textbf{Súper Importante}:Es conveniente que el último párrafo 
de la Introducción se utilice para resumir el objetivode la práctica.

Es la primera sección del texto propiamente dicho y constituye la presentación de una 
pregunta ¿Por qué se ha hecho este trabajo?

La introducción informa tres elementos muy importantes de la investigación: el 
propósito, la importancia y el conocimiento actual del tema.2,19,21 Requiere que el 
autor establezca el marco contextual en el que se inserta el problema que se va a 
resolver, qué es lo que se sabe acerca del asunto en cuestión, qué es lo que se 
desconoce y qué representaría desde el punto de vista científico, tecnológico, económico 
y social conocer lo que no se sabe. Esta construcción de lo general a lo particular 
concluye evidentemente con el objetivo, la hipótesis de trabajo o ambos, que son los 
elementos con los que debe finalizar este apartado.

La introducción debe basarse en las evidencias que aparecen en la literatura para 
establecer la originalidad, el interés y la lógica del problema que se va a resolver lo 
que se debe realizar con la máxima capacidad de síntesis.

Esta sección es una forma de atraer al lector y darle la mayor información posible.14,19 
No obstante debe ser breve y concisa porque con la abundancia de trabajos de revisión 
existentes, puede beneficiarse de lo expuesto en la revisión más reciente sobre el tema. 
Deben elegirse cuidadosamente las referencias que van a suministrar los antecedentes más 
importantes y citar a autores que hayan hecho trabajos previos relacionados con el 
artículo y que se consideren necesarios.

Debido a que con alguna frecuencia los artículos son leídos por personas que no 
pertenecen a la especialidad del autor, la Introducción es el lugar apropiado para 
situar y definir los términos o abreviaturas especializados que se vayan a utilizar.1

El texto debe ser claro y objetivo evitando la redundancia natural del idioma español y 
las figuras literarias.

Esta sección se debe redactar en \textbf{tiempo presente}

\section{Objetivo y planteamiento del problema}

Identificación del problema central de estudio y enfoque de la práctica.Importante:Esta  
sección  se  relaciona  con  el  procedimiento  experimental  presentado  en  el 
protocolo, pero no lo es del todo. Se trata que identifiques dentro de tu desarrollo 
experimental el sentido que tuvo la práctica para poder abordar el tema a estudiar.

El problema de investigación no tiene por qué ser una declaración, pero por lo menos debe 
dar a entender lo que estás tratando de encontrar.
Muchos autores prefieren colocar la declaración de tesis o hipótesis aquí, lo que es 
perfectamente aceptable, pero la mayoría la incluye en las últimas oraciones de la 
introducción, para brindar al lector una visión más completa.

\section{Materiales y método}
\label{sec:met}

Esta sección responde a la pregunta:¿Cómo se ha hecho el estudio?3, Tiene como meta 
describir minuciosamente, pero sin exageraciones ni redundancias, la forma en que se 
realizó el estudio.

Con la descripción detallada de los materiales y métodos se busca que los investigadores 
y lectores que accedan al artículo puedan reproducir el estudio y determinar la 
confiabilidad y la validez de sus resultados.

Este acápite puede estructurarse en los siguientes epígrafes:
\begin{description}
	\item[Diseño :]  Se describe el diseño del estudio detallando si se trata de un 
	ensayo 
	clínico aleatorio controlado, un estudio de casos y controles, o de cohorte, etc.
	\item[Población]  sobre la que se ha hecho el estudio : Describe el marco muestral y 
	cómo 
	se ha hecho su selección.
	\item[Entorno :]  Indica dónde se ha hecho el estudio: hospital, policlínico, 
	escuela, 
	etc.
	\item[Intervenciones:]  Se describen las técnicas empleadas, los tratamientos 
	farmacológicos, los aparatos e instrumentos utilizados, la tecnología empleada, etc. 
	Además, se deben especificar los fundamentos éticos del estudio y los aspectos éticos 
	que 
	se tuvieron en cuenta en caso de experimentos con humanos.
	\item[Análisis estadístico :]  Se deben describir las pruebas estadísticas empleadas 
	para 
	analizar los datos y especificar los programas estadísticos y las versiones empleadas.
\end{description}

Al describir los métodos de las investigaciones se deben aportar suficientes detalles 
para que un investigador competente pueda repetir la investigación.1,14 Si el método 
utilizado se ha publicado y es accesible a la mayoría de los lectores, es suficiente 
mencionar la fuente bibliográfica para no repetirlo. No obstante, podría describirse si 
es corto o si aparece en un trabajo difícil de conseguir. 

Si el método es nuevo, hay que proporcionar todos los detalles necesarios. Si es un 
método sustancialmente modificado se deben exponer las razones de su uso y evaluar sus 
limitaciones.

Cuando esta sección resulta larga por la cantidad o la complejidad de materiales y 
métodos utilizados, es conveniente usar subtítulos, por ejemplo: Condiciones generales, 
Tratamientos, Mediciones, etc., lo que ayuda a encontrar lo que se busca, a la vez que 
contribuye a evitar repeticiones. 

Cuando el artículo se somete al arbitraje, un buen árbitro leerá los Materiales y métodos 
detenidamente y si hay serias dudas sobre la posibilidad de repetir los experimentos, 
recomendará que el manuscrito sea rechazado, por asombrosos que sean los resultados.1

En este apartado se trata con elementos exactos y específicos, por lo que se debe cuidar 
la precisión del lenguaje y evitar el uso de palabras que puedan producir ambigüedad en 
la interpretación como: frecuentemente y periódicamente, que deben sustituirse por 
términos que expliquen exacta y claramente qué se hizo, cuándo se hizo y cómo se hizo. 
Tampoco se deben mezclar en esta sección algunos de los resultados.

La mayor parte de este apartado debe escribirse en pasado.



Ejemplo de ecuaciones

\begin{equation}\label{EQ:001}
	A = B+c
\end{equation}

\begin{eqnarray}
	E&=&mc^2,\label{equationa}
	\\
	E&=&mc^2,\label{equationb}
	\\
	E&=&mc^2,\label{equationc}
\end{eqnarray}


Ejemplo de sub-ecuación:
\begin{subequations}
	\label{allequations} % notice location
	\begin{eqnarray}
		E&=&mc^2,\label{equationa}
		\\
		E&=&mc^2,\label{equationb}
		\\
		E&=&mc^2,\label{equationc}
	\end{eqnarray}
\end{subequations}


\section{Desarrollo experimental}


En esta secci\'on deben describirse los materiales y métodos utilizados para el 
desarrollo de la práctica. Para describir el arreglo experimental, puede utilizarse una 
lista:
\begin{itemize}
	\item Elemento 1
	\item Elemento 2
	\item \ldots
\end{itemize}
o puede describirse en un párrafo cómo se formó el arreglo. Sin embargo, es de carácter 
obligatorio que el arreglo experimental se describa en un diagrama (no fotos).
Finalmente, en esta secci\'on debe describirse cómo se llevó a cabo el experimento.

\section{Resultados}
\label{Sec:res}

Conjunto  de  datos  obtenidos  experimentalmente  y  tratados  estadísticamente.  Usa  
tablas  paraorganizar y resumir los resultados\\\textbf{Importante}:I ncluye  solo  los  
datos  importantes  y  relevantes \cite{Berman1983},  pero  suficientes  para  
justificar  tus conclusiones. Usa ecuaciones, gráficos y figuras.\\\textbf{Súper  
	Importante}:
Si  cuentas  con  datos  teóricos  no  olvides  obtener  el  error  
experimental  (\%E). Esta información es muy relevante para justificar tus conclusiones 
\cite{Agarwal2001}.
Cuando construyas tablas no olvides:
\begin{enumerate}
	\item  Titulo de la tabla
	\item Distinción clara de celdas.
	\item Títulos de columnas.
	\item Notas al pie de la tabla
	\item Numeración de tablas
\end{enumerate}

Esta sección es la parte más importante del artículo y a menudo es también la más corta, 
especialmente si el acápite de Materiales y métodos que la precede y el de Discusión que 
le sigue, están bien redactados. El primer párrafo de este texto debe ser utilizado para 
resumir en una frase concisa, clara y directa, el hallazgo principal del estudio.

Este acápite se debe limitar a los datos que se vinculan con los objetivos del artículo, 
pues la inclusión de datos excesivos e innecesarios a la luz de los objetivos o de las 
hipótesis, solo demuestra que el autor carece de capacidad para discernir entre lo 
importante y lo irrelevante en el contexto de la finalidad del artículo.11 Se sugiere 
mencionar los hallazgos relevantes e incluso aquellos contrarios a la hipótesis, pues 
esto le dará seriedad y credibilidad al trabajo.

Los resultados se presentarán en el orden lógico y sucesivo en que fueron encontrados, de 
forma que sean comprensibles y coherentes por sí mismos.19 Ellos tienen que expresarse de 
manera clara y sencilla, porque representan los nuevos conocimientos que se están 
aportando a los lectores.

El uso de tablas y gráficos es una buena opción, siempre que se evite la redundancia, es 
decir la repetición con palabras de lo que resulta ya evidente al examinar estas formas 
de presentación de los resultados.

Las tablas y gráficos deben ser autoexplicativas, o sea, deben poder entenderse sin 
necesidad de leer el texto que les hace referencia.11 No se deben utilizar cuando los 
datos que se quieran presentar se puedan resumir en dos o tres párrafos dentro del 
texto.

Todas las tablas y los gráficos se citarán en el cuerpo principal de esta sección y se 
enumerarán en el orden en que aparecen en el texto.3,20 Su número debe ser el 
estrictamente necesario para ilustrar los resultados del estudio. Se recomienda usar 
gráficos como alternativa a las tablas con muchas entradas y no duplicar datos en los 
gráficos y tablas.

Esta sección debe ser escrita utilizando los \textbf{verbos en pasado} (se encontró, se 
observó, 
etc.).
\section{Discusión}
\label{Sec:Disc}

Es la sección más compleja de elaborar y organizar así como la más difícil de escribir. 
En ella se interpretan los resultados en relación con los objetivos originales e 
hipótesis y el estado de conocimiento actual del tema en estudio.

Los dos elementos centrales de la Discusión de un artículo son indicar, a juicio del 
autor, qué significan los hallazgos identificados en la sección de resultados y cómo 
estos se relacionan con el conocimiento actual.

En la Discusión, los resultados se exponen, no se recapitulan. Por tanto estos se pueden 
mencionar someramente antes de discutirlos pero sin repetirlos en detalle.

Se recomienda comenzar con un breve resumen de los principales resultados; a continuación 
exponer los posibles mecanismos o explicaciones de dichos hallazgos, compararlos y 
contrastarlos con los resultados de otros estudios relevantes, presentar las limitaciones 
del estudio, y argumentar las implicaciones de los resultados para futuras 
investigaciones y para la práctica clínica.

Debe realizarse la comparación de los hallazgos obtenidos con los resultados de 
investigaciones realmente comparables, así como con investigaciones que apoyan la 
hipótesis y también con aquellas que la contradicen.

Es necesario tener precaución con la discusión de resultados que no son significativos. 
Algunos autores presentan tales resultados, dicen que no son significativos y proceden a 
discutirlos como si lo fuesen.

No se debe prolongar este apartado innecesariamente citando trabajos relacionados o 
planteando explicaciones poco probables. Por otra parte, hay que considerar que una pobre 
discusión conlleva que el significado de los datos se oscurezca y que el artículo sea 
rechazado, aún teniendo datos sólidos.

Se deben exponer y comentar claramente, en lugar de ocultarlos, los resultados anómalos, 
dándoles una explicación lo más coherente posible o simplemente manifestando que esto es 
lo que se ha encontrado, aunque por el momento no tenga explicación. Si no lo hace el 
autor, seguro lo hará el editor.

Esta sección se redacta en presente: ``\textbf{estos datos indican que}'', porque los 
hallazgos del trabajo se consideran ya evidencia científica . 

\section{Conclusiones}
\label{Sec:Concl}
Al final de la discusión o en una sección separada, de acuerdo con las características de 
cada revista, se deben reflejar las conclusiones más significativas y la importancia 
práctica del estudio.

Las conclusiones son generalizaciones derivadas de los resultados y constituyen los 
aportes y las innovaciones del estudio realizado.19 Debido a que son producto de los 
resultados y la discusión, se debe evitar hacer afirmaciones rotundas y sacar más 
conclusiones de las que los resultados permitan.

La forma más simple de presentar las conclusiones es enumerándolas consecutivamente, 
aunque se puede optar por recapitular brevemente el contenido del artículo, mencionando 
someramente su propósito, los métodos principales, los datos más sobresalientes y la 
contribución más importante de la investigación, y evitar repetir literalmente el 
contenido del resumen.

Se sugiere no hacer conclusiones sobre los costos y beneficios económicos, a menos que el 
manuscrito incluya datos económicos con sus correspondientes análisis. Tampoco se deben 
hacer afirmaciones o alusiones a aspectos de la investigación que no se hayan llevado a 
término. 

La discusión puede incluir recomendaciones y sugerencias para investigaciones futuras, 
tales como métodos alternos que podrían dar mejores resultados, tareas que no se hicieron 
y que debieron hacerse y aspectos que merecen explorarse en las próximas investigaciones.

\section{Bibliografía}

Las referencias bibliográficas constituyen un grupo de datos precisos detallados para la 
identificación de una fuente documental impresa o no, de la cual se obtuvo la información.

En esta sección se detallarán los trabajos a los que se hizo referencia en el artículo y 
que deben ser numerados consecutivamente en el orden en que se mencionan por primera vez 
en el texto.

Debe existir siempre una correspondencia entre las citas que haya hecho en su trabajo y 
las que anexe en la literatura citada, ya que normalmente los lectores estarán 
interesados en verificar los datos que efectivamente se utilizaron para la investigación.

El error más frecuente en esta sección es transcribir incorrectamente algún dato de la 
cita, lo que dificultará su localización por parte del lector.

Las referencias cumplen dos funciones esenciales: testificar y autentificar los datos no 
originales del trabajo y proveer al lector de bibliografía referente al tema en cuestión.

Sólo se deben incluir como citas válidas artículos ya publicados en revistas científicas, 
artículos aceptados para publicación especificando que dicho trabajo se encuentra en 
prensa o en proceso de publicación; libros, capítulos de libros, tesis que formen parte 
de catálogos de bibliotecas y documentación disponible en internet.

La mayoría de las revistas no aceptan citas de comunicaciones personales, tesis no 
publicadas, resúmenes de presentaciones en congresos y manuscritos en preparación.

\bibliography{Bibliografia}% Produces the bibliography via BibTeX.


\appendix
\section*{Apendice 1}

\section*{Apendice 2}

\end{document}
%
% ****** End of file apssamp.tex ******